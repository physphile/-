\unchapter{Введение}

В настоящее время вкладываются большие усилия в развитие таких областей оптики, как интегральная фотоника, оптические коммуникации и вычисления. Ускоренному росту этих сфер способствуют открытия в области нанофотоники, изучающие новые технологии создания устройств управления светом. К числу таких технологий можно отнести и способы контроля направления распространения оптических пучков. Наиболее современные способы представляют собой полностью оптические методы, исследование которых активно продолжается. Важную роль в рамках этих исследований играет поиск взаимодействующих с излучением структур, которые позволяют решить поставленную задачу максимально эффективно. 

Данная работа посвящена разработке с помощью численных методов дифракционной решетки с переменным периодом, которая позволит непрерывно перестраивать углы дифракционных порядков при когерентном освещении за счет изменении фазы одного из источников. Были выполнены задачи оптимизации геометрических параметров решетки методом конечных разностей во временной области с целью определения максимальных углов отклонения и наибольшей относительной интенсивности дифракционных порядков. 